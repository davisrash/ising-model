% This is a report that abandons the ACM template
% The template can be readded at the end

% FORMAT REQUIREMENTS
% 500 words / One page single spaced
% 12 pt font

\documentclass[12pt]{article}

\usepackage{mathtools}
\usepackage{blindtext}

% \usepackage[margin=1in]{geometry}

\newcommand*{\eu}{e}
\newcommand*{\iu}{i}

\DeclarePairedDelimiter{\bra}{\langle}{\rvert}
\DeclarePairedDelimiter{\ket}{\lvert}{\rangle}
\DeclarePairedDelimiterX{\braket}[2]{\langle}{\rangle}
  {#1\,\delimsize\vert\,\mathopen{}#2}

\usepackage[hidelinks]{hyperref}

\title{Solving the Transverse-Field Ising Model on a Quantum Computer}
\author{David Basoco \and Jack Hetherington \and Davis Rash \and Tim Ross}
\date{October 4, 2023}

\begin{document}
  \maketitle

  % LIKELY ABANDON ABSTRACT
  % An abstract is already supposed to be ~250 words. In this case, it will
  % probably keep us from writing enough elsewhere.
  % \begin{abstract}
  %  This project investigated the ability to simulate the Ising model using quantum gates. The code was tested on both quantum simulators and the public IBM quantum computer. [Discuss Results]
  % \end{abstract}

  \section{Introduction}
  The transverse-field Ising Model is a quantum mechanical model of lattice sites with spin. In one dimension, the model can be described by the Hamiltonian
  \begin{equation}
    \label{eq:hamiltonian}
    H = \sum_{i = 1}^{n} \sigma_{i}^{x} \sigma_{i + 1}^{x}
        + \sigma_{1}^{y} \sigma_{2}^{z} \dotsm \sigma_{n - 1}^{z} \sigma_{n}^{y}
        + \lambda \sum_{i = 1}^{n} \sigma_{i}^{z},
  \end{equation}
  where \( \lambda \) is the transverse magnetic field strength. The second term maps periodic boundary conditions to fermionic degrees of freedom. Its effect vanishes for \( n \to \infty \) but will alter our result for \( n = 4 \).

  We simulated the Ising model on quantum simulators and the IBM [name] quantum computer. We implement the case \( n = 4 \). We conclude by comparing the results to the expected values for the Ising model % \cite{Cervera18}.

  \section{Quantum Gate Operations}
  To solve Eq.~\eqref{eq:hamiltonian} on a quantum computer, we first

  In order to accurately decompose the Ising model Hamiltonian we needed to take a decoupled Hamiltonian and convert it to the Ising model Hamiltonian. [Insert Hamiltonian conversion eqs.]

  \subsection{Jordan--Wigner Transform}
  Under the Jodan--Wigner transformation
  \begin{equation}
    \label{eq:jordan-wigner}
    c_{j}
      = \biggl( \prod_{l < j} \sigma_{l}^{z} \biggr)
        \frac{\sigma_{j}^{x} - \iu \sigma_{j}^{y}}{2},
  \end{equation}
  the Hamiltonian becomes
  \begin{equation}
    \label{eq:hamiltonian-fermion}
    H = \frac{1}{2}
        \sum_{i = 1}^{n} (c_{i + 1}^{\dagger} c_{i}^{\vphantom{\dagger}}
                          + c_{i}^{\dagger} c_{i + 1}^{\vphantom{\dagger}}
                          + c_{i}^{\dagger} c_{i + 1}^{\dagger}
                          + c_{i}^{\vphantom{\dagger}}
                            c_{i + 1}^{\vphantom{\dagger}})
        + \lambda \sum_{i = 1}^{n} c_{i}^{\dagger} c_{i}^{\vphantom{\dagger}}.
  \end{equation}
  Fortunately, the Jordan--Wigner transformation requires no gates to implement; it is a simple relabeling of the qubits. However, now any swapping must be done with the fermionic SWAP (fSWAP) gate
  \begin{equation}
    \label{eq:fswap}
    \mathrm{fSWAP}
      = \begin{pmatrix}
          1 & 0 & 0 & 0 \\
          0 & 0 & 1 & 0 \\
          0 & 1 & 0 & 0 \\
          0 & 0 & 0 & -1
        \end{pmatrix}.
  \end{equation}

  \subsection{Quantum Fourier Transform}

  \begin{equation}
    \label{eq:qft}
    b_{k}
      = \frac{1}{\sqrt{n}}
        \sum_{j = 1}^{n} \eu^{2 \pi \iu j k / n} c_{j}, \qquad
    k = -\frac{n}{2} + 1, \dotsc, \frac{n}{2}
  \end{equation}

  Although this is valid in general, we have \( n = 2^{m} \), permitting us to use the fast Fourier transform.

  \subsection{Bogoliubov Transform}
  \blindtext

  \section{Time Evolution}
  \blindtext

  \section{Expected Results}
  \blindtext

  \section{Simulation Results}
  \blindtext

  \section{Quantum Computer Results}
  \blindtext

  \section{Conclusion}
  \blindtext
\end{document}
