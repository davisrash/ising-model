% This is a report that abandons the ACM template
% The template can be readded at the end

% FORMAT REQUIREMENTS
% 500 words / One page single spaced
% 12 pt font

\documentclass[12pt]{article}

\usepackage{mathtools}
\usepackage{blindtext}
\usepackage{bbding}


% \usepackage[margin=1in]{geometry}

\newcommand*{\eu}{e}
\newcommand*{\iu}{i}

\DeclarePairedDelimiter{\bra}{\langle}{\rvert}
\DeclarePairedDelimiter{\ket}{\lvert}{\rangle}
\DeclarePairedDelimiterX{\braket}[2]{\langle}{\rangle}
  {#1\,\delimsize\vert\,\mathopen{}#2}

\usepackage[hidelinks]{hyperref}

\title{Solving the Transverse-Field Ising Model on a Quantum Computer}
\author{David Basoco \and Jack Hetherington \and Davis Rash \and Tim Ross}
\date{October 4, 2023}

\begin{document}
  \maketitle

  % LIKELY ABANDON ABSTRACT
  % An abstract is already supposed to be ~250 words. In this case, it will
  % probably keep us from writing enough elsewhere.
  % \begin{abstract}
  %  This project investigated the ability to simulate the Ising model using quantum gates. The code was tested on both quantum simulators and the public IBM quantum computer. [Discuss Results]
  % \end{abstract}

  \section{Introduction}
  The transverse-field Ising Model is a quantum mechanical model of lattice sites with spin. In one dimension, the model can be described by the Hamiltonian
  \begin{equation}
    \label{eq:hamiltonian}
    H = \sum_{i = 1}^{n} \sigma_{i}^{x} \sigma_{i + 1}^{x}
        + \sigma_{1}^{y} \sigma_{2}^{z} \dotsm \sigma_{n - 1}^{z} \sigma_{n}^{y}
        + \lambda \sum_{i = 1}^{n} \sigma_{i}^{z},
  \end{equation}
  where \( \lambda \) is the transverse magnetic field strength. The second term maps periodic boundary conditions to fermionic degrees of freedom. Its effect vanishes for \( n \to \infty \) but will alter our result for \( n = 4 \).

  We simulated the Ising model on quantum simulators and the IBM [name] quantum computer. We implement the case \( n = 4 \). We conclude by comparing the results to the expected values for the Ising model % \cite{Cervera18}.

  \section{Quantum Gate Operations}
  To solve Eq.~\eqref{eq:hamiltonian} on a quantum computer, we first

  In order to accurately decompose the Ising model Hamiltonian we needed to take a decoupled Hamiltonian and convert it to the Ising model Hamiltonian. We can obtain this by applying the unitary disentangling operator such that:
  \begin{equation}
    \tilde{H} = U_{dis}^{\dagger}HU_{dis},
  \end{equation}
  where H is the model Hamiltonian and $\tilde{H}$ is a non interacting Hamiltonian. For the case of the Ising Hamiltonian, the method to obtain the $U_{dis}$ is three fold. 
  \begin{enumerate}
   \item We first need to map spins to fermionic modes with the Jordan-Wigner transform. 
   \item Then we need to get the fermions in moment space by applying the Quantum Fourier transform. 
   \item Finally, we perform the Bogoliubov transform to decouple the modes in opposite momenta. 
  \end{enumerate}
  This gives us the three step recipe for out unitary disentangling operator:
  \begin{equation}
    U_{dis} = U_{JW}U_{QFT}U_{Bog}
  \end{equation}

  \subsection{Jordan--Wigner Transform}
  The Jordan-Wigner Transformation converts spin operators into the fermionic creation and annihilation operators.
  Under the Jodan--Wigner transformation
  \begin{equation}
    \label{eq:jordan-wigner}
    c_{j}
      = \biggl( \prod_{l < j} \sigma_{l}^{z} \biggr)
        \frac{\sigma_{j}^{x} - \iu \sigma_{j}^{y}} {2}, 
    c^{\dagger}_{j} = \frac{\sigma^{x}_{j}-\iu \sigmar^{y}_{j}}    
        {2}, \biggl( \prod_{l < j} \sigma^{z}_{l} \biggr)
  \end{equation}
  Then the Hamiltonian becomes
  \begin{equation}
    \label{eq:hamiltonian-fermion}
    H = \frac{1}{2}
        \sum_{i = 1}^{n} (c_{i + 1}^{\dagger} c_{i}^{\vphantom{\dagger}}
                          + c_{i}^{\dagger} c_{i + 1}^{\vphantom{\dagger}}
                          + c_{i}^{\dagger} c_{i + 1}^{\dagger}
                          + c_{i}^{\vphantom{\dagger}}
                            c_{i + 1}^{\vphantom{\dagger}})
        + \lambda \sum_{i = 1}^{n} c_{i}^{\dagger} c_{i}^{\vphantom{\dagger}}.
  \end{equation}
  Fortunately, the Jordan--Wigner transformation requires no gates to implement; it is a simple relabeling of the qubits. However, now any swapping must be done with the fermionic SWAP (fSWAP) gate
  \begin{equation}
    \label{eq:fswap}
    \mathrm{fSWAP}
      = \begin{pmatrix}
          1 & 0 & 0 & 0 \\
          0 & 0 & 1 & 0 \\
          0 & 1 & 0 & 0 \\
          0 & 0 & 0 & -1
        \end{pmatrix}.
  \end{equation}

  \subsection{Quantum Fourier Transform}
  Now we need to get the fermionic modes to momentum space with the quantum Fourier Transform.

  \begin{equation}
    \label{eq:qft}
    b_{k}
      = \frac{1}{\sqrt{n}}
        \sum_{j = 1}^{n} \eu^{2 \pi \iu j k / n} c_{j}, \qquad
    k = -\frac{n}{2} + 1, \dotsc, \frac{n}{2}
  \end{equation}

  Although this is valid in general, we have \( n = 2^{m} \), permitting us to use the fast Fourier transform.

  \subsection{Bogoliubov Transform}
  The final step is to decouple the modes that have opposite momentum. This will act over two qubit gates that represent opposite momenta.
  \begin{equation}
    B_{k}^{n} 
    = \begin{pmatrix}
        \cos(\frac{\theta_k}{n}) & 0 & 0 & i\sin(\frac{\theta_k}{n}) \\
        0 & 1 & 0 & 0 \\
        0 & 0 & 1 & 0 \\
        i\sin(\frac{\theta_k}{n}) & 0 & 0 & \cos(\frac{\theta_k}{n})
    \end{pmatrix},
  \end{equation}
  \begin{equation}
    \theta_k = \arccos(\frac{\lambda - \cos(\frac{2\pi k}{n})}{\sqrt{(\lambda - \cos(\frac{2\pi k}{n}))^2 + \sin^2(\frac{2\pi k}{n})}}).
  \end{equation}
  This returns our diagonal Hamiltonian:
    \begin{equation}
      \tilde{H} = H_{a} = \sum_{k=\frac{-n}{2}+1}^{\frac{n}{2}} \omega_{k}a_{k}^{\dagger}a_{k},
    \end{equation}
    where $\omega_{k} = \sqrt{(\lambda - \cos(\frac{2\pi k}{n}))^2 + \sin^2(\frac{2\pi k}{n})^2}$. This result is consistant with a transformation which mixes the two modes in accordance with
    \begin{equation}
      a_{k} = u_{k}b_{k}+iv_{k}c_{-k}^{\dagger},
    \end{equation}
    \begin{equation}
      a_{k}^{\dagger} = u_{k}c_{k}^{\dagger}+iv_{k}c_{-k}.
    \end{equation}

  \section{Time Evolution}
  \blindtext

  \section{Expected Results}
  \blindtext

  \section{Simulation Results}
  \blindtext

  \section{Quantum Computer Results}
  \blindtext

  \section{Conclusion}
  \blindtext
\end{document}
